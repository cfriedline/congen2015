\documentclass{article}
\usepackage{fullpage}
\usepackage{graphicx}
\usepackage{hyperref}

\title{ConGenOmics 2015 Winter Session}
\date{1/3/2015--7/3/2015}
\author{Christopher J. Friedline, Ph.D.\\Andrew J. Eckert, Ph.D.}

\begin{document}
\maketitle

\section*{Introduction}

This document is meant to serve as a guide for installing and configuring the
software required to carry out and build association study pipelines.  This
guide focuses on Python 2.7.x, IPython 2.x, and R 3.x.  By following along
with the installation procedures in this document, you will be able to 1)
install a standalone suite of software useful for your research at your own
institution and 2) work with the installed software at the B@G workshop.
Additionally, this guide will focus on OS X installation and will be largely
applicable to Linux and Unix systems, as well.  Windows users will be able to
work with the majority of the software outlined in this document, however, not
everything is likely to work.  Please consult with your High Performance
Computing group, or lab mates, about getting access to a Linux or OS X system
to carry out your analyses.

\section*{R}

By far the easiest way to get \texttt{R} installed is to download and install
RStudio from \url{http://www.rstudio.com}.  For systems to which you don't
have acess to a window system (i.e., X11), you can install R from source if it
is not already included in your distribution.  It's important to remember that
if you want to be able to use \texttt{R} functionality from other systems
(e.g., Python), then you need to enable access to it as a shared library when
you compile it.  This is done by adding a flag to the configure script,
\texttt{--enable-R-shlib}.  By default, packages that you get from package
managers such as \texttt{yum} and \texttt{apt}, or from
\url{http://r-project.org} already have this capability, so don't worry about
it.  You'll also want to make note of where \texttt{R\_HOME} is located on
your system, as it might be required by other packages (like \texttt{rpy2}).

\paragraph{Compiling R}
If you have to compile R for yourself, and you do not have \texttt{root} access
to your system (which is common), the following should work find, provided that
you have the \texttt{gcc} toolchain already installed (which you should).  If
you need a newer version for OS X, use MacPorts (\url{http://www.macports.org})
to manage it and save yourself a lot of time and heartache.  To compile
\texttt{R} on a system to which you do not have root access, do the following:

\begin{enumerate}
\item{Download the \texttt{R} source code from a CRAN mirror nearest you}
\item{\texttt{tar zxvf R-3.1.2.tar.gz} (or whatever version you download)}
\item{\texttt{cd R-3.1.2}}
\item{\texttt{./configure --prefix=\$HOME/R3 --enable-R-shlib}}
\item{\texttt{make}}
\item{\texttt{make install}}
\end{enumerate}

This process will install R into your home directory and the binary will be
located in \texttt{\$HOME/R3/bin}.  Make sure to add that to your path in your login
script. Additionally, the value of \texttt{R\_HOME} will be, at least in this
case, \texttt{\$HOME/R3/lib64/R}.

\paragraph{Important and useful packages} You should install these packages
which are helpful for association studies.  If you've not used \texttt{R}
before, you can install packages by starting the \texttt{R} environment and
entering the command \texttt{install.packages()}.

\begin{itemize}
\item{\texttt{ape}}
\item{\texttt{phangorn}}
\item{\texttt{lme4}}
\item{\texttt{SNPassoc}}
\item{\texttt{hierfstat}}
\item{\texttt{adegenet}}
\item{\texttt{seqinr}}
\item{\texttt{devtools}}
\end{itemize}

One confusing thing about \texttt{R} packages is that when you install them, you
must surround them with quotes (e.g., \texttt{install.packages("ape")}), but
when you load them into your \texttt{R} workspace, you do not use quotes (e.g.,
\texttt{library(ape)})

\section*{Python}

Installing from a scratch a new Python ecosystem can often involve myriad and
convoluted compilation of tens to hundreds of dependencies before even getting
to the libraries useful to your research.  To avoid this, I always recommend
installing the Anaconda Python distribution from Continuum.  You can access
the appropriate download for your system from
\url{http://continuum.io/downloads}.  As an added benefit, as an active
academic, you can get access to faster and more robust versions of some of the
core libraries.  See \url{https://store.continuum.io/cshop/academicanaconda} for
more information.

By default, the Anaconda distribution will install to your home directory in a
folder named \texttt{anaconda}.  I've not seen many occaisons to change this.
Once installed, you should choose the option to add the anaconda path to your
login scripts (e.g., .basrhc, .zshrc, etc.).  Upon logging out and in again
(or sourcing your login script) you should be able to run the \texttt{conda} 
command.

The \texttt{conda} command is the main command that you will use to both 
update/install software as well as create new python environments.  Conda
environments, similar in concept to virtualenv/venv, allow you to maintain 
project-specific python installations that contain only those libraries that 
are required.  This aids in usability for you, package isolation for your
software, and enables a layer of reprodicibility because you can easily 
idenfity what python packages and their versions are in use for your project
(via \texttt{pip freeze}).

\paragraph{Installing Anaconda}

Follow these steps to install and configure your own instance of Anaconda
Python.

\begin{enumerate}
\item{Download the install script from \texttt{continuum.io}}
\item{\texttt{sh Anaconda\_2.1...sh}}
\item{Accept license}
\item{Install to \texttt{$\sim$/anaconda}}
\item{Add to install path when prompted}
\item{Log out/in or source your login script}
\item{Update the conda software manager: \texttt{conda update conda}}
\item{Update the anaconda distribution: \texttt{conda update anaconda}}
\item{Update the base install with your required pacakges: \texttt{conda
install ipython ipython-notebook numpy scipy matplotlib pandas biopython
statsmodels setuptools pip}}
\item{Install other software that is not distributed with Anaconda:
\texttt{pip install rpy2}}
\end{enumerate}



\end{document}