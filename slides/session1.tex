%!TEX root = slides.tex

\section[Session 1]{Session 1: Evolutionary genetics of associations among
genotypes, phenotypes, and environments}

\begin{frame}
\centering
\includegraphics[width=0.8\textwidth]{sork}
\\
\cite[Adapted from][]{Sork:2013tb}
\end{frame}

\begin{frame}
\frametitle{Evoluationary quantitative genetics \citep{Walsh:2008}}
\begin{block}{Goals}
\begin{itemize}
\item{Understand the genetics and inheritance of complex traits}
\item{Nature and strength of evolutionary forces}
\item{Natural populations}
\end{itemize}
\end{block}

\begin{block}{Natural selection}
\begin{itemize}
\item{On traits}
\item{Heritability}
\item{Variation/Variance}
\item{Additive genetic variation}
\end{itemize}
\end{block}
\end{frame}

\begin{frame}
\frametitle{The data we have is from multiple sources}
\begin{block}{Our data}
\begin{itemize}
\item{A sampling of individuals}
\item{A sampling of populations made up of those individuals}
\item{Some genetic data from these individuals}
\item{Some phenotypic data about the individuals}
\item{Some data about the populations (e.g., location)}
\end{itemize}

\end{block}\end{frame}

\begin{frame}
\frametitle{There are several things we'd like to do with our data}

\begin{block}{Genetic architecture}
\begin{itemize}
\item{Ascertain a meaningful set of genetic variants which sufficient
discriminatory power}
\item{Use this genetic variation to understand something about the 
traits of the organism we care about}
\item{In other words, given a set of traits that we observe in nature, and as
scientists we find interesting, can we attribute variation in that trait with
variation in the genome}
\end{itemize}
\end{block}
\end{frame}



\begin{frame}
\frametitle{There is some math we need}
\begin{block}{}
\begin{equation}
\label{eqn:V_P}
V_P = V_G + V_E + V_{GE}
\end{equation}

\begin{equation}
\label{eqn:V_G}
V_G = V_A + V_D + V_I
\end{equation}

\begin{equation}
\label{eqn:V_A}
V_A = 2pq\alpha^2
\end{equation}

\begin{equation}
\label{eqn:alpha}
\alpha = a + d(q-p)
\end{equation}

\begin{equation}
\label{eqn:a}
a = \frac{G_{AA}-G_{aa}}{2}
\end{equation}

\begin{equation}
\label{eqn:d}
d = G_{Aa} - \frac{G_{AA}+G_{aa}}{2}
\end{equation}

\end{block}{}
\end{frame}

\begin{frame}
\frametitle{Variance components}
\begin{block}{Equation \ref{eqn:V_P}: $V_P = V_G + V_E + V_{GE}$}
\begin{itemize}
\item{$V_P$: Phenotypic variance ($\sigma^2_P$)}
\item{$V_G$: Genetic variance}
\item{$V_E$: Environmental variance}
\item{$V_{GE}$: Genetic + environmental variance}
\end{itemize}
\end{block}
\end{frame}

\begin{frame}
\frametitle{Genetic variance}
\begin{block}{Equation \ref{eqn:V_G}: $V_G = V_A + V_D + V_I$}
\begin{itemize}
\item{$V_G$: Total genetic variance}
\item{$V_A$: Additive genetic variance}
\item{$V_D$: Dominance genetic variance}
\item{$V_I$: Epistatic genetic variance (I = interaction)}
\end{itemize}
\end{block}
\end{frame}

\begin{frame}
\frametitle{Additive gentic variance}
\end{frame}

\begin{frame}
\frametitle{Dominance gentic variance}
\end{frame}

\begin{frame}
\frametitle{Interaction gentic variance}
\end{frame}

\begin{frame}
\frametitle{Heritability}
\end{frame}

\begin{frame}
\frametitle{Heritability}
\begin{block}{An example}
\end{block}
\end{frame}



